\documentclass[12pt,a4paper]{article}
\usepackage[utf8]{inputenc}
\usepackage{polski}
\usepackage[polish]{babel}
\usepackage{amsmath}
\usepackage{amsfonts}
\usepackage{amssymb}
\usepackage{graphicx}
\usepackage[export]{adjustbox}
\usepackage{wrapfig}
\usepackage{caption}


\title{1. Fale podłużne w ciałach stałych}
\date{18 października 2017}	
\author{
	Zespół 3: Górski Paweł, Sozańska Ada\\
	EAIiIB Informatyka, Rok II
}
\renewcommand{\baselinestretch}{1.5}
\captionsetup[figure]{labelformat={default},name={\bfseries Rys.}}
\captionsetup[table]{labelformat={default},name={\bfseries Tab.}}

\begin{document}
\maketitle
% WPROWADZENIE
\section{Wprowadzenie}
Celem tego doświadczenia jest wyznaczenie w sposób doświadczalny przybliżonej wartości modułu Younga dla ciał stałych, wykonanych z różnych materiałów. 

Moduł Younga jest to wielkość charakteryzująca własności sprężyste ciała stałego i podatność materiału na odkształcenia podłużne przy rozciąganiu, ściskaniu lub zginaniu. Prawo Hooke'a definiuje nam zależność modułu Younga od innych wielkości:
\begin{equation}
	E = \frac{\sigma}{\varepsilon},
	\label{eq:hooke}
\end{equation}

gdzie $\sigma$ to naprężenie normalne ciała, a $\varepsilon$ to odkształcenie względne ciała.

Rozważając pręt o gęstości $\rho$ i prędkości fali rozchodzącej się w pręcie $\upsilon$ oraz korzystając z prawa Hooke'a (równanie \ref{eq:hooke}), jesteśmy w stanie podać następującą zależność:
\begin{equation}
	E = \rho \upsilon^2.
\end{equation}

W rozważanym pręcie fale padająca i odbita interferują ze sobą, tworząc falę stojącą. Znając częstotliwość fali $f$ oraz jej długość $\lambda$ jesteśmy w stanie obliczyć prędkość fali w danym ciele:
\begin{equation}
	\upsilon = f\lambda
\end{equation}

Rozważamy również, że te pręty są obustronnie otwarte lub obustronnie zamknięte. W takim przypadku możemy przedstawić długość fali stojącej jako $\lambda = 2l$, gdzie $l$ jest długością pręta. Biorąc to pod uwagę, jesteśmy w stanie przedstawić formułę na moduł Younga, w której każda wielkość jest mierzalna:
\begin{equation}
	E = 4\rho l^2 f^2
\end{equation}

Fale dźwiękowe w pręcie można przybliżyć za pomocą złożenia harmonicznych drgań sinusoidalnych, a wykorzystując FFT (\emph{Fast Fourier Transform}) możemy otrzymać odpowiadające im częstotliwości. 

Częstotliwość najniższego tonu jest nazywana częstotliwością podstawową, a kolejne nazywane są częstotliwościami harmonicznymi, będącymi wielokrotnościami częstotliwości podstawowej.
\begin{equation}
	f_k = kf_0,~\textrm{gdzie}~k = 2, 3, 4 \ldots
\end{equation}

% WYKONANIE ĆWICZENIA
\pagebreak
\section{Wykonanie ćwiczenia}
W celu wykonania doświadczenia wykorzystaliśmy:
\begin{itemize}
	\item Komputer stacjonarny z mikrofonem,
	\item Oprogramowanie \emph{Zelscope},
	\item Zestaw 7 prętów (o różnych kształtach i z różnych materiałów),
	\item Suwmiarka,
	\item Miarka w rolce,
	\item Młotek,
	\item Dwie wagi elektroniczne o różnych dokładnościach.
\end{itemize}

Doświadczenie rozpoczęliśmy od zmierzenia i zważenia próbek prętów (lub samych prętów, gdy nie było im odpowiadających próbek) w celu wyznaczenia gęstości materiałów. Część pomiarów zostało wykonanych za pomocą suwmiarki z dokładnością $0,05~\mathrm{mm}$, inne (dłuższe) zostały zmierzone za pomocą miarki o działce elementarnej równej $1~\mathrm{mm}$. Do zważenia większości próbek użyliśmy wagi z dokładnością $0,01~\mathrm{g}$, natomiast jeden pręt (ze względu na jego wielkość) musiał być zważony wagą o mniejszej dokładności równej $1~\mathrm{g}$.

TU JESZCZE NIE WIEMY JAK ZROBILIŚMY POMIARY CZĘSTOTLIWOŚCI :( SMUTEK.
\pagebreak
% OPRACOWANIE DANYCH POMIAROWYCH
\section{Opracowanie danych pomiarowych}
\subsection{Pomiary}
Zmierzona długość wahadła wynosi $l = 341$ mm. Niepewność pomiaru długości wahadła $l$ wynosi $u(l) = 1$ mm, ponieważ jest to najmniejsza działka elementarna wykorzystanego narzędzia.

\begin{table}[!ht]
	\caption{Wartości pomiarów okresów wahadła dla stałej długości}
	\begin{center}
		\begin{tabular}{r|r|r|r}
			\hline
			\multicolumn{1}{c|}{L.p.} & \multicolumn{1}{c|}{liczba okresów $k$} & \multicolumn{1}{c|}{czas $t$ dla $k$ okresów [s]} & \multicolumn{1}{c}{okres $T_i = t/k$ [s]} \\ \hline \hline
			1 & 20 & 22,61 & 1,130 \\
			2 & 20 & 22,88 & 1,144 \\
			3 & 20 & 22,64 & 1,132 \\
			4 & 20 & 22,78 & 1,139 \\
			5 & 20 & 22,77 & 1,138 \\
			6 & 20 & 22,88 & 1,144 \\
			7 & 20 & 23,09 & 1,154 \\
			8 & 20 & 23,07 & 1,153 \\
			9 & 20 & 23,15 & 1,157 \\
			10 & 20 & 23,24 & 1,162 \\ \hline
		\end{tabular}
	\end{center}
	\label{tab:tab1}
\end{table}
Wartość średnia okresu $T$ dla wyników pomiaru okresu $T_i$ (Tab. \ref{tab:tab1}) jest równa $T = 1,145$. Niepewność pomiaru wartości okresu wyrażona wzorem:
\begin{equation}
	u(T) = \sqrt{\frac{\sum_{i=1}^{10}(T_i - T)^2}{n(n-1)}}
\end{equation}
wynosi $u(T) = 3,4$ ms.

Poniżej znajdują się wyniki pomiarów (Tab. \ref{tab:tab2}) przy zmiennej długości wahadła potrzebne do wykonania regresji liniowej.
\begin{table}[!ht]
	\caption{Wartości pomiarów okresów wahadła dla zmiennej długości}
	\begin{center}
		\begin{tabular}{r|r|r|r|r|r}
			\hline
			\multicolumn{1}{c|}{L.p.} & \multicolumn{1}{c|}{$l$ [mm]} & \multicolumn{1}{c|}{k} & \multicolumn{1}{c|}{$t$ [s]} & \multicolumn{1}{c|}{$T_i$ [s]} & \multicolumn{1}{c}{$T_{i}^{2} ~[\textrm{s}^2]$} \\ \hline \hline
			1 & 439 & 20 & 26,2 & 1,310 & 1,716 \\
			2 & 403 & 20 & 24,9 & 1,245 & 1,550 \\
			3 & 366 & 20 & 23,68 & 1,184 & 1,401 \\
			4 & 327 & 20 & 22,55 & 1,127 & 1,270 \\
			5 & 285 & 20 & 21,16 & 1,058 & 1,119\\
			6 & 241 & 20 & 19,32 & 0,9660 & 0,9331 \\
			7 & 194 & 20 & 17,87 & 0,8935 & 0,7983 \\
			8 & 153 & 20 & 15,19 & 0,7595 & 0,5768 \\
			9 & 108 & 20 & 13,21 & 0,6605 & 0,4362 \\
			10 & 63 & 20 & 10,34 & 0,5170 & 0,2672 \\ \hline
		\end{tabular}
	\end{center}
	\label{tab:tab2}
\end{table}

%\begin{table}[htbp]
%	\caption{Wartości pomiarów okresów wahadła dla zmiennej długości}
%	\begin{center}
%		\begin{tabular}{r|r|r|r|r|r|r|r}
%			\hline
%			\multicolumn{1}{c|}{L.p.} & \multicolumn{1}{c|}{$l$ [mm]} & \multicolumn{1}{c|}{k} & \multicolumn{1}{c|}{$t$ [s]} & \multicolumn{1}{c|}{$T_i$ [s]} & \multicolumn{1}{c|}{$T_{i}^{2} ~[\textrm{s}^2]$} & \multicolumn{1}{c|}{$g~[\frac{\textrm{m}}{\textrm{s}^2}]$} & \multicolumn{1}{c}{$u(g)~[\frac{\textrm{m}}{\textrm{s}^2}]$} \\ \hline \hline
%			1 & 439 & 20 & 26,2 & 1,310 & 1,716 & 10,9 & 0,057 \\
%			2 & 403 & 20 & 24,9 & 1,245 & 1,550 & 10,26 & 0,061 \\
%			3 & 366 & 20 & 23,68 & 1,184 & 1,401 & 10,3 & 0,065 \\
%			4 & 327 & 20 & 22,55 & 1,127 & 1,270 & 10,16 & 0,069 \\
%			5 & 285 & 20 & 21,16 & 1,058 & 1,119 & 10,05 & 0,074 \\
%			6 & 241 & 20 & 19,32 & 0,9660 & 0,9331 & 10,19 & 0,083 \\
%			7 & 194 & 20 & 17,87 & 0,8935 & 0,7983 & 9,59 & 0,088 \\
%			8 & 153 & 20 & 15,19 & 0,7595 & 0,5768 & 10,47 & 0,116 \\
%			9 & 108 & 20 & 13,21 & 0,6605 & 0,4362 & 9,77 & 0,135 \\
%			10 & 63 & 20 & 10,34 & 0,5170 & 0,2672 & 9,3 & 0,192 \\ \hline
%		\end{tabular}
%	\end{center}
%	\label{tab:tab2}
%\end{table}

\subsection{Analiza wyniku dla stałej długości wahadła}
Wartość przyspieszenia ziemskiego $g$ dla długości wahadła $l$ oraz okresowi wahadła $T$ wynosi:
\begin{equation}
	\begin{split}
		&l = 0,341~\textrm{m},~~T = 1,145~\textrm{s}, \\ 
		&g = 4\pi^2\frac{l}{T^2} \approx 10,26~\frac{\textrm{m}}{\textrm{s}^2}.
	\end{split}
\end{equation}

Niepewność złożona pomiaru wartości przyspieszenia ziemskiego wyliczona, wykorzystując prawo przenoszenia niepewności wynosi:
\begin{equation}
	u_c(g) = 0,18~\frac{\textrm{m}}{\textrm{s}^2}.
\end{equation}
Niepewność rozszerzona pomiaru wartości przyspieszenia ziemskiego wynosi:
\begin{equation}
	U(g) = 0,36~\frac{\textrm{m}}{\textrm{s}^2}.
\end{equation}

Wartość tabelaryczna przyspieszenia ziemskiego w Krakowie wynosi $g_0 = 9,811~\frac{\textrm{m}}{\textrm{s}^2}$.

\begin{equation}
	|g - g_0| = 0,37 > U(g) = 0,36~[\frac{\textrm{m}}{\textrm{s}^2}]
	\label{eq:gconst}
\end{equation}

Wynik wskazuje na to, że mógł zostać popełniony błąd systematyczny podczas pomiarów okresów wahadła (żadne dane nie odbiegają od siebie w znaczący sposób). Prawdopodobnie wynika on z braku synchronizacji osób przeprowadzających doświadczenie.


\subsection{Analiza wyniku dla zmiennej długości wahadła}
\begin{figure}[!ht]
	\centering

	\caption{Wykres zależności okresu od długości wahadła}
	\label{fig:img2}
\end{figure}
\pagebreak
Na wykresie (Rys. \ref{fig:img2}) widzimy funkcję okresu wahadła od jego długości. Zależność ta nie jest zależnością liniową (wykres jest lekko pochylony). Dopiero zależność kwadratu okresu wahadła od jego długości przypomina funkcję liniową. 

\begin{figure}[!ht]
	\centering

	\caption{Wykres zależności kwadratu okresu od długości wahadła}
	\label{fig:img3}
\end{figure}


Wykorzystując metodę najmniejszych kwadratów przypasowaliśmy następującą prostą:
\begin{equation}
	\begin{split}
	&y = ax,~~ a = 3,887,\\ 
	&u(a) = 0,023.
	\end{split}
\end{equation}

Korzystając ze wzoru:
\begin{equation}
	g = \frac{4\pi^2}{a}
\end{equation}
wyliczamy wartość przyspieszenia ziemskiego $g = 10,15~\frac{\textrm{m}}{\textrm{s}^2}$.

Niepewność przyspieszenia ziemskiego wyliczona tą metodą jest równa $u_c(g) = 0,061~\frac{\textrm{m}}{\textrm{s}^2}$.
Porównując z wartością tabelaryczną jak poprzednio mamy:
\begin{equation}
	|g - g_0| = 0,27 > U(g) = 0,122~[\frac{\textrm{m}}{\textrm{s}^2}].
	\label{eq:gvar}
\end{equation}
Ponownie otrzymana wartość przyspieszenia ziemskiego $g$ nie jest zgodna z wartością tabelaryczną.
\section{Wnioski}
Mimo zmniejszenia czynnika ludzkiego (10 pomiarów 20 okresów) oraz wyliczenia średniej ze zmierzonych wartości w celu zminimalizowania niepewności, wyniki otrzymane przy pomocy obu metod nie są zgodne z wartością tabelaryczną dla Krakowa.

Dla naszych danych, metoda druga (wykorzystująca regresję liniową) cechuje się większą dokładnością (równanie \ref{eq:gvar}) aniżeli metoda pierwsza (równanie \ref{eq:gconst}), której niepewność otrzymanego wyniku jest ponad trzykrotnie większa od niepewności metody drugiej.

Różnice powstałe między tymi dwoma metodami są spowodowane błędem systematycznym nie wykrytym podczas pomiarów okresu wahadła lub nieprawidłowym przeprowadzeniem doświadczenia. Drastyczny wpływ na wynik mógł mieć zbyt duży kąt odchylenia wahadła (większy niż $3^\circ$), bowiem wtedy aproksymacja, którą wykorzystujemy ma większy błąd (zwiększenie z $3^\circ$ do $15^\circ$ powoduje ok. dwudziestokrotnie większy błąd).
\end{document}