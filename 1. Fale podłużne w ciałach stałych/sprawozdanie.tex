\documentclass[12pt,a4paper]{article}
\usepackage[utf8]{inputenc}
\usepackage{polski}
\usepackage[polish]{babel}
\usepackage{amsmath}
\usepackage{amsfonts}
\usepackage{amssymb}
\usepackage{graphicx}
\usepackage[export]{adjustbox}
\usepackage{wrapfig}
\usepackage{caption}
\usepackage{breqn}

\numberwithin{equation}{section}

\renewcommand{\baselinestretch}{1.5}
\captionsetup[figure]{labelformat={default},name={\bfseries Rys.}}
\captionsetup[table]{labelformat={default},name={\bfseries Tab.}}

\newcommand*{\captionsource}[2]{%
	\caption[{#1}]{%
		#1%
		\\\hspace{\linewidth}%
		\textbf{Żródło:} #2%
	}%
}



\title{1. Fale podłużne w ciałach stałych}
\date{18 października 2017}	
\author{
	Zespół 3: Górski Paweł, Sozańska Ada\\
	EAIiIB Informatyka, Rok II
}

\begin{document}
\maketitle
% WPROWADZENIE
\section{Wprowadzenie}
Celem tego doświadczenia jest wyznaczenie w sposób doświadczalny przybliżonej wartości modułu Younga dla ciał stałych, wykonanych z różnych materiałów. 

Moduł Younga jest to wielkość charakteryzująca własności sprężyste ciała stałego i podatność materiału na odkształcenia podłużne przy rozciąganiu, ściskaniu lub zginaniu. Prawo Hooke'a definiuje nam zależność modułu Younga od innych wielkości:
\begin{equation}
	E = \frac{\sigma}{\varepsilon},
	\label{eq:hooke}
\end{equation}

gdzie $\sigma$ to naprężenie normalne ciała, a $\varepsilon$ to odkształcenie względne ciała.

Rozważając pręt o gęstości $\rho$ i szybkości fali rozchodzącej się w pręcie $\upsilon$ oraz korzystając z prawa Hooke'a (równanie \ref{eq:hooke}), jesteśmy w stanie podać następującą zależność:
\begin{equation}
	E = \rho \upsilon^2.
	\label{eq:young}
\end{equation}

W rozważanym pręcie fale padająca i odbita interferują ze sobą, tworząc falę stojącą. Znając częstotliwość fali $f$ oraz jej długość $\lambda$ jesteśmy w stanie obliczyć szybkość fali w danym ciele:
\begin{equation}
	\upsilon = f\lambda.
	\label{eq:v}
\end{equation}

Przyjmijmy również, że te pręty są obustronnie otwarte lub obustronnie zamknięte. W takim przypadku możemy przedstawić długość fali stojącej jako $\lambda = 2l$, gdzie $l$ jest długością pręta. Biorąc to pod uwagę, jesteśmy w stanie przedstawić formułę na moduł Younga, w której każda wielkość jest mierzalna:
\begin{equation}
	E = 4\rho l^2 f^2.
\end{equation}

Fale dźwiękowe w pręcie można przybliżyć za pomocą złożenia harmonicznych drgań sinusoidalnych, a wykorzystując FFT (\emph{Fast Fourier Transform}) możemy otrzymać odpowiadające im częstotliwości. 

Częstotliwość najniższego tonu jest nazywana częstotliwością podstawową, a kolejne nazywane są częstotliwościami harmonicznymi, będącymi wielokrotnościami częstotliwości podstawowej.
\begin{equation}
	f_k = kf_0,~\textrm{gdzie}~k = 2, 3, 4 \ldots
	\label{eq:f_k}
\end{equation}

% WYKONANIE ĆWICZENIA
\pagebreak
\section{Wykonanie ćwiczenia}
W celu wykonania doświadczenia wykorzystaliśmy:
\begin{itemize}
	\item Komputer stacjonarny z mikrofonem,
	\item Oprogramowanie \emph{Zelscope},
	\item Zestaw 7 prętów (o różnych kształtach i z różnych materiałów),
	\item Suwmiarkę,
	\item Miarkę w rolce,
	\item Młotek,
	\item Dwie wagi elektroniczne o różnych dokładnościach.
\end{itemize}

Doświadczenie rozpoczęliśmy od zmierzenia i zważenia próbek prętów (lub samych prętów, gdy nie było im odpowiadających próbek) w celu wyznaczenia gęstości materiałów. Część pomiarów zostało wykonanych za pomocą suwmiarki z dokładnością $0,05~\mathrm{mm}$, inne (dłuższe pręty) zostały zmierzone za pomocą miarki o działce elementarnej równej $1~\mathrm{mm}$. Do zważenia większości próbek użyliśmy wagi z dokładnością $0,01~\mathrm{g}$, natomiast jeden pręt (ze względu na jego wielkość) musiał być zważony wagą o mniejszej dokładności równej $1~\mathrm{g}$.

Następnie do jednego końca pręta przyłożony został mikrofon tak, aby po uderzeniu młotkiem w drugi koniec zostały zarejestrowane harmoniczne tego drgania. Za pomocą programu \emph{Zelscope} (po zastosowaniu FFT) z wykresu zostały odczytane częstotliwości dla kolejnych harmonicznych z niepewnością oszacowaną na 20~Hz. Powyższa procedura została powtórzona dla czterech prętów (stalowego, aluminiowego, mosiężnego i miedzianego).

\pagebreak
% OPRACOWANIE DANYCH POMIAROWYCH
\section{Opracowanie danych pomiarowych}
\subsection{Gęstości materiałów}

Zmierzone zostały wymiary oraz masy próbek (lub samego pręta w przypadku braku próbki) dla każdego z czterech prętów.

Próbka stalowa miała kształt prostopadłościanu o wymiarach:
\begin{equation}
	\begin{split}
	 	&a = 14,20~\textrm{mm},~~b = 14,00~\textrm{mm}~~\textrm{i}~~c = 19,80~\textrm{mm}, \\
		&u(a) = u(b) = u(c) = 0,05~\textrm{mm}.
	\end{split}
\end{equation}
Niepewność pomiaru objętości tej próbki została wyznaczona korzystając z prawa propagacji niepewności:
\begin{align}
	&V = abc, \\
	&u(V) = \sqrt{(bc\cdot u(a))^2 + (ac\cdot u(b))^2 + (ab\cdot u(c))^2}.
\end{align}

Próbka aluminiowa miał kształt walca o średnicy $d$ i wysokości $l$:
\begin{equation}
	\begin{split}
		&d = 5,00~\textrm{mm},~~l = 440~\textrm{mm},\\
		&u(d) = 0,05~\textrm{mm},~~u(l) = 1~\textrm{mm}.
	\end{split}
\end{equation}
Niepewność pomiaru objętości tej próbki została wyznaczona korzystając z prawa propagacji niepewności:
\begin{align}
&V = d^2 l \frac{\pi }{4},  \label{eq:v_cyl} \\
&u(V) = \sqrt{\Bigg(d l \frac{\pi}{2} \cdot u(d)\Bigg)^2 + \Bigg(d^2 \frac{\pi}{4} \cdot u(l)\Bigg)^2}  \label{eq:uv_cyl}.
\end{align}

Próbka mosiężna miała kształt walca o średnicy $d$ i wysokości $l$:
\begin{equation}
\begin{split}
&d = 5,90~\textrm{mm},~~l = 312~\textrm{mm}, \\
&u(d) = 0,05~\textrm{mm},~~u(l) = 1~\textrm{mm}.
\end{split}
\end{equation}
Objętość próbki została policzona ze wzoru (\ref{eq:v_cyl}), a niepewność pomiaru objętości ze wzoru (\ref{eq:uv_cyl}).

Pręt miedziany nie posiadał próbki, dlatego też wymiary i masa zostały zmierzone bezpośrednio. Miał on kształt rury o średnicy zewnętrznej $d_{zew}$, średnicy wewnętrznej $d_{wew}$ i wysokości $l$:
\begin{equation}
	\begin{split}
		&d_{zew} = 18,15~\textrm{mm},~d_{wew} = 15,50~\textrm{mm},~~l = 1802~\textrm{mm},\\
		&u(d_{zew}) = u(d_{wew}) = 0,05~\textrm{mm},~~u(l) = 1mm.
	\end{split}
\end{equation}
Objętość rury została policzona ze wzoru:
\begin{equation}
	V = (d_{zew}^2 - d_{wew}^2) l \frac{\pi }{4}.
\end{equation}
Niepewność pomiaru objętości tej rury została wyznaczona korzystając z prawa propagacji niepewności:
\begin{equation}
	u(V) = \sqrt{\Bigg(\frac{\partial V}{\partial d_{zew}} \cdot u(d_{zew})\Bigg)^2 + \Bigg(\frac{\partial V}{\partial d_{wew}} \cdot u(d_{wew})\Bigg)^2  + \Bigg(\frac{\partial V}{\partial l} \cdot u(l)\Bigg)^2},
\end{equation}
gdzie
\begin{equation}
	\begin{split}
		&\frac{\partial V}{\partial d_{zew}} = d_{zew} l \frac{\pi}{2} - 1,\\[5pt]
		&\frac{\partial V}{\partial d_{wew}} = 1 - d_{wew} l \frac{\pi}{2},\\[5pt]
		&\frac{\partial V}{\partial l} = (d_{zew}^2 - d_{wew}^2) \frac{\pi }{4}.
	\end{split}
\end{equation}

\pagebreak
Gęstości materiałów wyznaczone zostały za pomocą wzoru:
\begin{equation}
	\rho = \frac{m}{V},
\end{equation}
gdzie $m$ jest masą badanej próbki, a $V$ objętością tej próbki.
Niepewność pomiaru gęstości została wyznaczona na mocy prawa przenoszenia niepewności:
\begin{equation}
	u(\rho) = \sqrt{\Bigg(\frac{u(m)}{V}\Bigg)^2 + \Bigg(-m \frac{u(V)}{V}\Bigg)^2}.
\end{equation}
Wyniki dla poszczególnych materiałów zostały zestawione w tabeli \mbox{(Tab. \ref{tab:tab1})}.

\begin{table}[!ht]
	\caption{Objętości wraz z gęstościami dla badanych materiałów}
	\begin{center}
		\begin{tabular}{r|c|c|c|c|c|c}
			\hline  \rule{0pt}{18pt} 
			Materiał & $m$~[g] & $u(m)$~[g] & $V~\big[\textrm{cm}^3\big]$ & $u(V)~\big[\textrm{cm}^3\big]$ & $\rho~\Big[\frac{\textrm{kg}}{\textrm{m}^3}\Big]$ & $u(\rho)~\Big[\frac{\textrm{kg}}{\textrm{m}^3}\Big]$ \\[3pt] \hline \hline
			Stal & 30,862 & 0,001 & 3,936 & 0,022 & 7840 & 44  \\
			Aluminium & 23,883 & 0,001 & 8,63 & 0,17 &  2764 & 55  \\
			Mosiądz & 74,536 & 0,001 &  8,53 & 0,14 & 8738 & 150   \\
			Miedź & 760 & 1 & 126,2 & 3,3 & 6022  & 161 \\ \hline
		\end{tabular}
	\end{center}
	\label{tab:tab1}
\end{table}

\subsection{Analiza składowych harmonicznych}

Dla częstotliwości podstawowej, długość rozchodzącej się fali stojącej $\lambda$ wynosi $\lambda = 2l$, gdzie $l$ to długość pręta. Z zależności między częstotliwościami składowych harmonicznych (\ref{eq:f_k}), wzoru na szybkość fali (\ref{eq:v}) i faktu, iż szybkość ta jest stała dla danego ośrodka wynika, że dla każdej kolejnej składowej harmonicznej wartość ta jest $n$ razy mniejsza, gdzie $n$ oznacza numer harmonicznej.

W tabeli poniżej zawarte są wyniki otrzymane przy pomocy programu \emph{Zelscope}.

\pagebreak
\begin{table}[!ht]
	\caption{Częstotliwości składowych harmonicznych oraz odpowiadające im szybkości fali dla poszczególnych materiałów}
	\centering
	\begin{adjustbox}{height=0.448\textheight}
		\begin{center}
			\begin{tabular}{l||c|c|c|c|c|c}
				\multicolumn{5}{c}{\hspace{3.25cm}\bfseries Stal} & \multicolumn{2}{c}{$l = 1,8$~m} \\ \hline
				Nr harmonicznej & 1 & 2 & 3 & 4 & 5 & 6  \\
				$f$ [Hz] & 1410 & 2900 & 4310 & 5720 & 7120 & 8600  \\
				$\lambda$ [m] & 3,600 & 1,800 & 1,200 & 0,900 & 0,720 & 0,600  \\ 
				$f_0$ [Hz] & 1410 & 1450 & 1436 & 1430 & 1424 & 1433  \\
				$\upsilon$ $\Big[\frac{\textrm{m}}{\textrm{s}}\Big]$ & 5076 & 5220 & 5172 & 5148 & 5126 & 5160  \\[1pt] \hline
				\multicolumn{5}{c}{\hspace{3.25cm}\bfseries Aluminium} & \multicolumn{2}{c}{$l = 1$~m} \\ \hline
				Nr harmonicznej & 1 & 2 & 3 & 4 & 5 & 6  \\
				$f$ [Hz] & 2435 & 4970 & 6842 & 9560 & 11340 & 12370 \\
				$\lambda$ [m] & 2,000 & 1,000 & 0.666 & 0,500 & 0,400 & 0,333  \\ 
				$f_0$ [Hz] & 2435 & 2485& 2280& 2390& 2268& 2061  \\
				$\upsilon$ $\Big[\frac{\textrm{m}}{\textrm{s}}\Big]$ & 4870 & 4970 & 4561 & 4780 & 4536 & 4123 \\[1pt] \hline
				\multicolumn{5}{c}{\hspace{3.25cm}\bfseries Mosiądz} & \multicolumn{2}{c}{$l = 1$~m} \\ \hline
				Nr harmonicznej & 1 & 2 & 3 & 4 & 5 & 6  \\
				$f$ [Hz] & 1690 & 3470 & 5160 & 6840 & 8630 & 12000 \\
				$\lambda$ [m] & 2,000 & 1,000 & 0.666 & 0,500 & 0,400 & 0,333  \\ 
				$f_0$ [Hz] & 1690 & 1735 & 1720 & 1710 & 1726 & 2000  \\
				$\upsilon$ $\Big[\frac{\textrm{m}}{\textrm{s}}\Big]$ & 3380 & 3470 & 3440 & 3420 & 3452 & 4000 \\[1pt] \hline
				\multicolumn{5}{c}{\hspace{3.25cm}\bfseries Miedź} & \multicolumn{2}{c}{$l = 1,8$~m} \\ \hline
				Nr harmonicznej & 1 & 2 & 3 & 4 & 5 & 6  \\
				$f$ [Hz] & 1180 & 2160 & 3240 & 4280 & 5260 & 6200  \\
				$\lambda$ [m] & 3,600 & 1,800 & 1,200 & 0,900 & 0,720 & 0,600  \\ 
				$f_0$ [Hz] & 1180 & 1080 & 1080 & 1070 & 1052 & 1033  \\
				$\upsilon$ $\Big[\frac{\textrm{m}}{\textrm{s}}\Big]$ & 4248 & 3888 & 3888 & 3852 & 3787 & 3720 \\ \hline
			\end{tabular}
		\end{center}
	\end{adjustbox}
	\label{tab:tab2}
\end{table}
\pagebreak

Analizując dane z tabeli (Tab. \ref{tab:tab2}) można zauważyć, że dla aluminium oraz mosiądzu wartości $f_0$ dla szóstej składowej harmonicznej znacznie odbiegają od pozostałych. Podobna sytuacja zachodzi w przypadku pierwszej częstotliwości dla pręta miedzianego. Prawdopodobnie został popełniony błąd gruby, dlatego też wyniki te odrzucamy. Następnie dla pozostałych danych wyznaczamy średnie wartości szybkości $\upsilon$ i częstotliwości podstawowej $f_0$. Dodatkowo dla wyznaczonych wartości gęstości $\rho$ i szybkości $\upsilon$ obliczamy moduł Younga $E$ każdego z omawianych materiałów. Obliczoną wartość $E$ porównujemy z wartościami tabelarycznymi $E_0$ odpowiadających im materiałów (zaczerpniętymi z \emph{www.engineeringtoolbox.com}).

Niepewność dla szybkości $\upsilon$ wyznaczamy stosując prawo przenoszenia niepewności dla wzoru (\ref{eq:v}):
\begin{equation}
u(\upsilon) = \sqrt{(f \cdot u(\lambda))^2 + (\lambda \cdot u(f))^2}.
\end{equation}

Identycznie postępujemy dla wzoru (\ref{eq:young}) na moduł Younga:
\begin{equation}
u(E) = \sqrt{(\upsilon^2 \cdot u(\rho))^2 + (2\rho\upsilon \cdot u(\upsilon))^2}.
\end{equation}

Dla niepewności $u(\lambda) = 1$~mm, $u(f_0) = 20$~Hz i $u(\rho)$ w tabeli (Tab. \ref{tab:tab1}) obliczamy niepewności $u(\upsilon)$ i $u(E)$ według powyższych wzorów. Wszystkie wyniki zawarte są w poniżej tabeli.

\begin{table}[!ht]
	\caption{Obliczona i tabelaryczna wartość modułu Younga}
	\centering
	\begin{adjustbox}{width=1\textwidth}
	\begin{center}
		\begin{tabular}{r|c|c|c|c|c|c}
			\hline  \rule{0pt}{18pt} 
			Materiał & $f_0$~[Hz] & $\upsilon$~$\Big[\frac{\textrm{m}}{\textrm{s}}\Big]$ & $u(\upsilon)$~$\Big[\frac{\textrm{m}}{\textrm{s}}\Big]$ & $E~[\textrm{GPa}]$ & $u(E)~[\textrm{GPa}]$ & $E_0~[\textrm{GPa}]$\\ \hline \hline
			Stal & 1430 & 5150 & 72 & 207,9 & 5,9 & 200 \\
			Aluminium & 2371 & 4743 & 40 & 62,1 & 1,6 & 69  \\
			Mosiądz & 1716 & 3432 & 40 & 102,9 & 2,9 & 102 - 125  \\
			Miedź & 1063 & 3827 & 72 & 88,2 & 4,0 & 117  \\ \hline
		\end{tabular}
	\end{center}
	\end{adjustbox}
	\label{tab:tab3}
\end{table}

Dla stali wartość modułu Younga jest zgodna w granicy błędu, ponieważ:
\begin{equation}
	\Big|E_0 - E\Big| = 7,9 < U(E) = 11,8~[\textrm{GPa}].
\end{equation}

Dla aluminium obliczona wartość modułu Younga nie jest zgodna z wartością tabelaryczną, gdyż:
\begin{equation}
\Big|E_0 - E\Big| = 6,9 > U(E) = 3,2~[\textrm{GPa}].
\end{equation}

Dla mosiądzu wartość modułu Younga zawiera się w przedziale dopuszczalnym dla tego materiału:
\begin{equation}
	102 < E = 102,9 < 125~[\textrm{GPa}].
\end{equation}

Dla miedzi obliczona wartość modułu Younga nie jest zgodna z wartością tabelaryczną, bowiem:
\begin{equation}
\Big|E_0 - E\Big| = 28,8 > U(E) = 8,0~[\textrm{GPa}].
\end{equation}

\pagebreak
\section{Wnioski}

Dzięki pomiarom masy i objętości próbek (oraz rury), byliśmy w stanie wyznaczyć odpowiadające im gęstości. Następnie przy pomocy programu \emph{Zelscope} otrzymaliśmy częstotliwości składowych harmonicznych co pozwoliło nam wyznaczyć szybkość rozchodzenia się fali w prętach, a w konsekwencji obliczyć wartość modułu Younga.

Otrzymane w ten sposób wartości dla stali oraz mosiądzu zgadzają się z odpowiadającymi im wartościami tabelarycznymi.

Niestety, mimo odrzucenia błędu grubego dla aluminium i miedzi wartość modułu Younga dla tych materiałów nie jest zgodna z wartością tabelaryczną. W przypadku aluminium można podejrzewać, że został popełniony błąd systematyczny, ponieważ wartości częstotliwości podstawowej $f_0$ cechowały się stosunkowo dużym rozrzutem. Natomiast w przypadku miedzi, wyznaczona gęstość nie była zgodna z wartością tabelaryczną $\rho_0 = 8940~\Big[\frac{\textrm{kg}}{\textrm{m}^3}\Big]$:
\begin{equation}
\Big|\rho_0 - \rho\Big| = 2918 > U(\rho) = 322~\Big[\frac{\textrm{kg}}{\textrm{m}^3}\Big],
\end{equation}
co może sugerować, że rura ta nie była wykonana z miedzi lub pomiary masy i jej rozmiaru mogły zostać źle wykonane. Nie można wykluczyć też, że błędne było założenie, iż średnica wewnętrzna tej rury była stała na całej długości.

\end{document}
